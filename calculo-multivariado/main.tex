\documentclass{article}

\usepackage[spanish]{babel}
\usepackage{mathtools}
\usepackage{bm}
\usepackage{amssymb, amsmath, amsbsy}

\newcommand{\uveci}{{\bm{\hat{\textnormal{\bfseries\i}}}}}
\newcommand{\uvecj}{{\bm{\hat{\textnormal{\bfseries\j}}}}}
\DeclareRobustCommand{\uvec}[1]{{%
  \ifcsname uvec#1\endcsname
     \csname uvec#1\endcsname
   \else
    \bm{\hat{\mathbf{#1}}}%
   \fi
}}
\newcommand{\tab}[1][1cm]{\hspace*{#1}}

\setlength{\topmargin}{-20mm}

\author{Jean Carlos Santoya Cabrera}
\begin{document}
    \title{Ejercicios cálculo multivariado}
    \maketitle
    \begin{flushleft}
        1) Encuentre la función de la posición de la partícula con parámetro t, dados: $\vec{r}(0) = (1, 2, 3)$, $\vec{r}(c) = (4, 1, 4)$,
         $\|\vec{v}(0)\| = 2$ y $\vec{a}(t) = 3\uvec{i} -\uvec{j} + \uvec{k}$.\\
        \tab[3.2mm] Solución:\\
        \text{- Primero hallamos la velocidad inicial:}\\
        \begin{equation}
            \uvec{v}_0 = direccion \cdot rapidez
        \end{equation}
        \begin{equation}
            \uvec{v}_0 = \frac{ \Delta \vec{r} }{ \|\Delta \vec{r}\| } \cdot 2
        \end{equation}
        \begin{equation}
            \uvec{v}_0 = (\frac{6}{\sqrt{11}}, -\frac{1}{\sqrt{11}}, \frac{1}{\sqrt{11}})
        \end{equation}

        \text{- Ahora hallamos la velocidad de la partícula:}\\
        \begin{equation}
            \uvec{v}(t) = velocidadInicial + aceleracion \cdot t
        \end{equation}
        \begin{equation}
            \uvec{v}(t) = \uvec{v}_0 + \vec{a} t
        \end{equation}
        \begin{equation}
            \uvec{v}(t) = (\frac{6}{\sqrt{11}} + 3t, -\frac{1}{\sqrt{11}} - t, \frac{1}{\sqrt{11}} + t)
        \end{equation}

        \text{- Ahora hallamos la posición de la partícula dado t:}\\
        \begin{equation}
            \uvec{r}(t) = velocidadInicial + aceleracion \cdot t
        \end{equation}
        \begin{equation}
            \uvec{r}(t) = \uvec{r}_0 + \vec{v} t
        \end{equation}
        \begin{equation}
            \uvec{r}(t) = (\frac{6}{\sqrt{11}}t + 3t^2, -\frac{t}{\sqrt{11}} - t^2, \frac{t}{\sqrt{11}} + t^2)
        \end{equation}
    \end{flushleft}
    
\end{document}